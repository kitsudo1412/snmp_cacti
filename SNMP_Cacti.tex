\documentclass[12pt,oneside,a4paper]{article}
\usepackage[english]{babel}
\usepackage[utf8x]{vietnam}
% package for including graphics with figure-environment
\usepackage{graphicx}
\usepackage{amsmath,amssymb,exscale,eucal,amsthm, mathrsfs}

%\DeclareMathSizes{12}{15}{9}{9}
\usepackage[table,xcdraw]{xcolor}
\usepackage{tabularx}

\usepackage[top=3.5cm, bottom=3cm, left=2cm, right=2cm] {geometry}
% colors for hyperlinks
% colored borders (false) colored text (true)
\usepackage{wrapfig}
\usepackage{color}
%\usepackage{caption}
% package for bibliography
%\usepackage[round, sort, numbers]{natbib}
% package for header
\usepackage[automark]{scrpage2}
\pagestyle{scrheadings}
\ihead[]{Tìm hiểu về SNMP và công cụ giám sát mạng Cacti - Đào Tuấn Anh, Vũ Thanh Tùng}
\ohead[]{\today}
\cfoot[]{\pagemark} 
\setheadsepline[122mm]{0.3mm}

\theoremstyle{definition}
\newtheorem{theorem}{Định lý}[section]
\newtheorem{lemma}{Bổ đề}[section]
\newtheorem{procedure}{Thủ tục}[section]
\newtheorem{definition}{Định nghĩa}[section]
\newtheorem{statement}{Mệnh đề}[section]

\usepackage{enumitem}

\newlist{steps}{enumerate}{1}
\setlist[steps,1]{
  label={Bước~\arabic*.},
  leftmargin=*,
  align=left,
  labelsep=10mm,
  itemindent=\dimexpr\labelsep+\labelwidth+7pt\relax
}

\begin{document}
		
		\addtolength{\leftskip}{5em} \addtolength{\rightskip}{2em}
		
		\begin{wrapfigure}[5]{l}{5em}\centering
			\includegraphics[width=0.1\textwidth]{bkhn.jpg}
		\end{wrapfigure}
		
		\noindent  \\
		 \\
		 \\
		Trường Đại học Bách Khoa Hà Nội \\
		Viện Toán ứng dụng và Tin học
		
		\addtolength{\leftskip}{-5em} \addtolength{\rightskip}{-2em}% Restore margins

		\title{\Huge{Báo cáo bài tập lớn}\\ \large{Môn: Thiết kế, cài đặt và quản trị mạng máy tính}\\}
	
	
	
	\vspace{1cm}
	
	\author{Đề tài: Tìm hiểu về SNMP và công cụ giám sát mạng Cacti}
	
	\vspace{1cm}
	
	% name of the course and module
	\date{
	\vspace{1cm}
	\begin{table}[h]
		\centering
		\begin{tabular}{rl}
			\large Sinh viên thực hiện: & \large Đào Tuấn Anh (MSSV: 20121178)\\
                    						& \large Vũ Thanh Tùng (MSSV: 20124917)
		\end{tabular}
	\end{table}	
	\vspace{1cm}
	\begin{table}[h]
		\centering
		\begin{tabular}{rl}
			\large Giáo viên hướng dẫn: & \large ThS. Nguyễn Tuấn Dũng\\
		\end{tabular}
	\end{table}
	\vspace{1cm}
	Hà Nội, \today
	}
	\global\let\newpagegood\newpage
	\global\let\newpage\relax\maketitle
	\global\let\newpage\newpagegood
	
	\setlength{\parindent}{0pt}

	\newpage
	\tableofcontents
	\newpage
	

\section*{Lời mở đầu}
\addcontentsline{toc}{section}{\protect\numberline{}Lời mở đầu}%
Sự phát triển không ngừng nghỉ của công nghệ thông tin đã tạo ra những thay đổi lớn trong cơ sở hạ tầng, lực lượng sản xuất, tính chất lao động, cấu trúc kinh tế và cả cách thức quản lý trong các lĩnh vực của xã hội. Một trong những yếu tố quyết định, cũng là một nhu cầu tất yếu là việc trao đổi thông tin giữa các thiết bị thông qua mạng máy tính. Sự tăng nhanh về kích cỡ của các hệ thống mạng đã và đang đưa ra thách thức không nhỏ cho người quản trị trong việc theo dõi và quản lý.\\

Do đó, các hệ thống giám sát an toàn mạng ra đời, đóng vai trò quan trọng, không thể thiếu trong hạ tầng công nghệ thông tin của các cơ quan, đơn vị, tổ chức. Hệ thống này cho phép thu thập, chuẩn hóa, lưu trữ và phân tích tương quan toàn bộ các sự kiện được sinh ra trong hệ thống mạng của tổ chức. Từ đó người quản trị có thể dễ dàng đưa ra phân tích, thống kê, cảnh báo, nắm bắt thực trạng tài nguyên, đồng thời ngăn chặn các mối nguy hiểm cho hệ thống.\\

Nội dung dưới đây có đề cập đến giao thức quản lý mạng SNMP (Simple Network Management Protocol). SNMP là một tập hợp các giao thức không chỉ cho phép kiểm tra các thiết bị mạng như router, switch hay server có đang vận hành mà còn hỗ trợ vận hành các thiết bị này một cách tối ưu, ngoài ra SNMP còn cho phép quản lý các thiết bị mạng từ xa. Bên cạnh đó, báo cáo còn đề cập đến Cacti, một hệ thống quản lý tài nguyên mạng mã nguồn mở. Phần mềm này đáp ứng nhu cầu quản lý mạng một cách toàn diện với nhiều tính năng linh hoạt vượt trội.\\

Em xin chân thành cảm ơn thầy Nguyễn Tuấn Dũng đã tận tình truyền đạt kiến thức trong suốt môn học. Tuy nhiên trong khoảng thời gian cho phép, việc nghiên cứu và trình bày không thể tránh khỏi nhầm lẫn thiếu sót, em mong nhận được sự giúp đỡ đóng góp của thầy cô và bạn bè để bài báo cáo được hoàn thiện hơn.\\

\textit{Sinh viên}
Đào Tuấn Anh - Vũ Thanh Tùng
\newpage

\section{Tìm hiểu về giao thức SNMP}
\subsection{Tổng quan về giám sát mạng (Network Monitoring)}
\subsection{Giao thức quản lý mạng SNMP (Simple Network Management Protocol}
\subsubsection{Giới thiệu giao thức SNMP}

\section{Công cụ giám sát mạng Cacti}
\subsection{Giới thiệu về công cụ giám sát mạng Cacti}
\subsection{Kiến trúc phần mềm Cacti}
\subsection{Cài đặt Cacti trên Windows}
\subsection{Ứng dụng Cacti trong giám sát mạng}

\newpage
\section*{Tài liệu tham khảo}
\addcontentsline{toc}{section}{\protect\numberline{}Tài liệu tham khảo}%
%\bibliographystyle{ksfh_nat}
%	\bibliography{references} % expects file "references.bib"
\begin{enumerate}
\item{Wikipedia, "Simple Network Management Protocol", https://en.wikipedia.org/wiki/SNMP}
\item{Wikipedia, "Network monitoring", https://en.wikipedia.org/wiki/Network\_monitoring}
\item{TBT - VNCERT, "Tổng quan về hệ thống giám sát mạng", http://www.vncert.gov.vn/}
\item{Diệp Thanh Nguyên, "SNMP Toàn tập", https://sites.google.com/site/snmptoantap/home}
\end{enumerate}
	

\end{document}